% http://pre14.deviantart.net/7a89/th/pre/i/2011/289/4/8/the_blue_silence_by_koaltaitemaunga-d4czpeo.png
## Places of Power

``Can you feel the peace of this grove, Bae'win?  The generations of
Druids who have tended to these woods, to the world?  Their souls may
have returned to the planes, but the outer planes are close here.  So
does the power of their will, a will of preservation, protect this
grove.  It is the perfect place for you to learn our ways.''

``There's magic in everyone.  It normally takes a lifetime of training
to do anything with it, but it's there.  Places like this, it's a bit
closer to the surface.  You just have to reach out and... *Whumpf,
flicker*... heh.  And there it is.''

``Too much magic was used here too quickly.  It lingers, tearing at
the fabric of reality.  But it is not the real threat, apprentice.
For as many wizards died during the wars of the Third Era, many more
have died here in the centuries since.  What killed them was a refusal
to remain humble.  We are servants of magic, child, not its master.
You would do well to remember that, lest your pride allow these
energies to rip you asunder... Now, bring me my reagents.  We have
work to do.''

### A title

Places of power exist throughout the world.  While the
nature of each varies, all such places offer both power and danger to
those who would use them.

This homebrew suggests a guideline for using these places of power
with a series of skill checks.  However, the skill check that is
appropriate will vary depending on the nature of the place of power.
Likewise, the spells such a place can empower will be determined my
the place of power's nature.  We will refer to the spells, attribute,
and skill associated to a place of power as Domain Spells, the Domain
Attribute, and Domain Skill.



> Possible Places of Power and associated domains
>##### Power Domains
| Site                             | Spell List | Attribute | Skill    |
|:--------------------------------:|:----------:|:---------:|:--------:|
| A convergence of ley lines       |  Wizard    |  Int      | Arcana   |
| An ancient shrine                |  Cleric    |  Wis      | Religion |
| A gateway to your patron's plane |  Warlock   |  Cha      | Religion |
| A place of wild magic            |  Sorcerer  |  Cha      | Arcana   |

Areas of great magical power can be the boon of the clever and bane of
the reckless.  These areas may exist at a convergence of ley lines, at
the site of great magical battles, or at a place where a {\it Wish}
spell has exposed the weave of magic in the Material Plane.  At these
places, a creature has a chance to reach out to the magic of the
location, powering spells not of their own strength, but of the
world's.

How an individual reacts to a place of magical power depends
inherently on that individual's relationship with magic.  A wizard's
or sorcerer's understanding of magic differs fundamentally from the
powers granted to warlocks and clerics by otherworldly beings.  A
place of magical power is more yielding to spellcasters.  Yet, it may
not be accessible to all creatures with the {\it Spellcasting}
feature, but rather only to those creatures with a feature of an
appropriate kind.  A convergence of ley lines in a sacred grove may
only be accessible to a Druid, whereas the power of an ancient and
oft-used sacrificial altar may only be accessible to a Warlock.
Conversely, the exposed edges of the Weave caused by a {\it Wish}
spell might be visible to all spellcasters.

In this way, a place of power is {\it attuned} to a particular type of
magic.  Spells that appear on the appropriate spell lists are {\it
  attuned spells}.

A creature may spend a bonus action to perform any number of the
following actions.  Multiple actions may be attempted, but all must be
declared together.  Only one attempt may be made for any individual
action, except as noted.  A creature may not, for instance, attempt
{\it Unanchor Magic} twice in one bonus action.

All checks are made using {\it Intelligence (Arcana)} against the
given DC.

Any effect that calls for the spell's level treats cantrips as level
zero.

For any instance in which failure deals damage, failure by rolling a
natural 1 should deal damage as if from a critical hit.

  
  
  If multiple effects are attempted, attempt them in the order listed
  here.  Any failure that causes the spell to fizzle ends the action,
  denying all further checks.  Additionally, many actions directly
  influence a spell cast on this turn.  If failure during this bonus
  action causes the spell to fail, the action that would have cast the
  spell is also lost.  Such are the dangers of wielding volatile
  magics.
  


  \section*{Bonus Actions in Places of Power}

  Magic is so readily available to those who would reach for it, even
  the uninitiated may bend it.  A player may use a bonus action to
  reach towards the weave of magic and attempt one of the following:

\action{Unanchor Magic}{15} A creature attempts to unanchor a spell
sustained by {\it Magical Sustain} (below).  \success The spell ends
immediately.  \failure The spellcaster unanchors themselves, floating
slightly above the ground.  The spellcaster's movespeed is reduced to
5 ft for 1d4 rounds.

\action{Magical Intuition}{10+} A creature reaches into the
surrounding magic, attempting to gain a magical intuition for
spellcasting.  Select a spell attuned to the place of power.  Increase
the DC of this check by the spell's level multiplied by 5.  If the
creature has an atuned spellcasting level, reduce the DC of this check
by that amount.  This action may only be taken once per short or long
rest.  \success The chosen spell is considered known and prepared
until the creature's next short or long rest.  \failure Torrents of
magic flood the creature's mind.  The creature takes 2d6 psychic
damage, with an additional 1d6 multiplied by the spell's level.  If
not otherwise incapacitated, the creature is then stunned for 1d4
rounds.  For higher level spells, multiply the time stunned by the
spell's level.  \note This action, with enough time and sufficient
resources, would allow a dangerously quick expansion of spellbooks.
Such sites are sought for and battled over fiercely.

\action{Magical Prowess}{20+} \attuned A creature attempts to cast a
spell of the first level or higher using the ambient power instead of
one of its own spell slots.  Increase the DC of this check by 5 for
each level past the first of the slot that would be consumed.  This
may allow a creature without the spellcasting feature to cast a spell
of first level or higher.  This action may be taken twice if used in
conjunction with {\it Mirror Spell}.  \success The spell does not
consume a spell slot when cast.  \failure If the creature fails this
check by 5 or more, or if the creature has no remaining spell slots of
the correct level, the spell fizzles.  Otherwise, the spell resolves
normally, consuming the appropriate spell slot.
  
  \section*{When used in conjunction with attuned spells}

\action{Magical Sustain}{20} \attuned A spellcaster attempts to weave
a spell requiring concentration into the ambient magic to sustain the
spell.  \success The spellcaster does not maintain concentration on
this spell.  The DM makes a secret roll (1d20 - 10, with minimum 1) to
determine the number of rounds the spell persists.  This spell is not
controlled by the spellcaster, nor can it be freely dismissed.  Any
effect that requires an action (i.e. later damage from {\it
  Witchbolt}) or other commands (i.e. direction for {\it Bigby's
  Hand}) chosen by the DM.

\failure The spell slot used to cast this spell is consumed.  The
spell fizzles with no effect.  The DM makes a secret roll (1d20 - 10,
with minimum 1); the spellcaster cannot cast a spell requiring
concentration for this many rounds.

\action{Empower Spell}{15 + spell's cast level} \attuned Drawing from
ambient magics, a spell's strength is redoubled.  \success The spell
resolves as if cast one level higher than the slot consumed.  \failure
The spellcaster takes 1d6 + 1d6 per spell level of a damage.  If the
spell cast deals damage, the caster receives damage of the same type.
If the spell does not deal damage, the DM chooses the type (Psychic or
Force recommended).  If the DC was failed by less than 5, the spell is
resolves, empowered as above.  If the DC was failed by 5 or more, the
spell fizzles, consuming the spell slot.

\action{Mirror Spell}{15 + spell's cast level} \attuned

  {\it This action may only be used in conjunction with spells that
    require an attack roll.  If the spell would make multiple attacks,
    only one attack is mirrored.}

  {\it This action not be used with spells that require
    concentration.}

  A creature guides bends ambient magic to mirror its own spells.
  \success The spellcaster casts this spell again, attacking a new
  target.  The spellcaster must expend spell slots to power both
  spells (that is, two spell slots if only this action was used, but
  only one if this is used with a successful {\it Magical Prowess}).
  The spells must be cast at the same level.  A spellcaster may
  consume a higher-level spell slot if necessary, but doing so does
  not raise the level of the spell cast.  \failure The spellcaster
  casts this spell again, but must target themselves.  If this check
  was failed by less than 5, the spellcaster may instead choose to let
  both spells fizzle.  (Spell slots are still consumed and this
  round's action is lost.)

\action{Mutate Spell}{10} \attuned \success The damage type of the
spell cast may be changed to another type.  \failure The caster takes
2d6 damage of the intended damage type.  If the check is failed by 5
or more, the spell resolves with the original damage type.  Otherwise,
the damage type is changed as intended.

\end{multicols}
\end{document}


-----

Some thoughts:

  Some groups might not do well with the whole "Do as many of these as
  you want in one turn" mentality.  My group tends to be pretty
  low-risk; they might try to intuit a cantrip before a fight, but
  definitely not during one.  pretty conservative when it comes to
