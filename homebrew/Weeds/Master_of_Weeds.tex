% \documentclass{article}
\documentclass{report}

\usepackage{amssymb,amsmath,amsthm}
\usepackage[left=1in,top=1in,right=1in,nohead]{geometry}
\usepackage{hyperref} % makes ToC clickable
\usepackage{eso-pic} % allows background picture patterning
\usepackage{graphicx} % \includegraphics

%\usepackage{chemarr}
%\usepackage{tikz}
%\usepackage{marginnote} % allows margin notes inside extra environments
%\usepackage{algorithm,algorithmic}
%\usepackage{float}

%\usepackage[left=1in,top=1in,right=1in,nohead]{geometry}
%\theoremstyle{remark}
%\newtheorem{lemma}{Lemma}
%\usepackage{xcolor}
%\usepackage{   caption}
%\usepackage{subcaption}

%\usepackage[draft]{todonotes}
%\floatstyle{boxed}%p{figure}
%\restylefloat{figure}


% This folder:

\newcommand{\Edges}{\mathcal{E}}
\newcommand{\Verts}{\mathcal{V}}
\newcommand{\ER}{Erd\H{o}s-R\'enyi}
\newcommand{\Fi}{\F^{(i)}}
\newcommand{\eiv}{\nu}
\newcommand{\eif}{\upsilon}
\newcommand{\imp}[2]{I_{ #1 }^{(#2) }}
\newcommand{\impl}[3]{I_{ #1 #3 }^{( #2 )} }
\newcommand{\PM}[1]{P\left(\mbox{#1}\right)}
\newcommand{\av}{\mu}
\newcommand{\bx}{\mathbf{x}}
\newcommand{\bksl}{\backslash}
\newcommand{\bvp}{\bar{v}_p}
\newcommand{\Pths}{\mathcal P}
\newcommand{\Ind}{\mathcal I}
\newcommand{\dprod}{\displaystyle\prod}
\newcommand{\comment}[1]{ { {\color{red} #1 } } }
\newcommand{\state}{\STATE\ \ }

% Comment macros


% Sets
\newcommand{\Ps}{\mathcal{P}}
\newcommand{\N}{\mathcal{N}}
\newcommand{\E}{\mathcal{E}}
\newcommand{\I}{\mathcal{I}}
\newcommand{\F}{\mathcal{F}}
\newcommand{\Set}{\mathcal{S}}
\newcommand{\D}{\mathcal{D}}
\newcommand{\V}{\mathcal{V}}

% Common
\newcommand{\ds}{\displaystyle}
% \newcommand{\bit}{\begin{itemize}}
\newcommand{\eit}{\end{itemize}}
\newcommand{\lp}{\left(}
\newcommand{\rp}{\right)}
\newcommand{\ddt}{\frac{d}{dt}}
\newcommand{\floor}[1]{\left\lfloor #1 \right\rfloor}

% Vector/Matrix
%\newcommand{\bf}{\textbf }

\newcommand{\zmat}{\left[ 0 \right] }
\newcommand{\dv  }[1]{\left[ \begin{array}{c}    #1 \end{array}\right]}
\newcommand{\dtwm}[1]{\left[ \begin{array}{cc}   #1 \end{array}\right]}
\newcommand{\dtm }[1]{\left[ \begin{array}{ccc}  #1 \end{array}\right]}
\newcommand{\dfm }[1]{\left[ \begin{array}{cccc} #1 \end{array}\right]}

\usepackage{color} % Color red DM only stuff
\usepackage{fancyhdr} % Footnote the date
\newif\ifdm
% \dmtrue

\newcommand{\wherein}[1]{\textbf{ \LARGE Wherein} \begin{itemize} #1 \end{itemize}} 
\newcommand{\witem}[1]{\item[$\cdot$] {\em #1} }
\newcommand{\dmonly}[1]{\ifdm { \color{red} #1 } \else { } \fi}
\newcommand{\HS}{H\'arom-s\'al}

\begin{document}

\dmonly{ {\LARGE DM COPY} }

\AddToShipoutPictureBG{
  \includegraphics{parchment-paper-light-texture.png}
}


\section*{Prestige Class: Master of Weeds}

The Master of Weeds works to unlock the full potential of poisons through careful study and experimentation.
Given the nature of their work, they are often found with titles of Royal Poisoner or Master Assassin, finding positions either exempt from or far outside the law of the land.


\paragraph*{Prerequisites}:
\begin{itemize}
\item  Intelligence 14
\item  Constitution 14
\item  Proficiency with Poisoner's Kit
\item  Complete a special task: Build a labratory.
\end{itemize}
    


\begin{tabular}{rll}
{\bf Level} & {\bf Features} & {\bf Personal Labratory} \\&&{\bf Crafting Speed} \\
1st & Poison Expertise            & 10gp \\
    & Poisonous Innovation  &\\
    & Personal Labratory    &\\
2nd & Cautious Cook           & 25gp  \\
3rd & Poison Pin           & 50gp     \\
4th & Death Eater           & 100gp   \\
5th & Poison Savant           & 200gp \\
\end{tabular}

\section*{Class Features}

\paragraph*{Hit Points}
1d8 rogue?

\paragraph*{Proficiencies}
None


\paragraph*{Poison Expertise}
  At 1st level, your proficency bonus is doubled when using the poisoner's kit.
  Additionally, you are considered proficient in Intelligence (Medicine) for the purpose of extracting poison from a creature, if you were not proficient already.

\paragraph*{Personal Labratory}
At 1st level, when using downtime to create a poison in your personal labratory, you can make progress in 10gp increments instead of 5gp.
This bonus increases to 15gp increments at 2nd level, 20gp at 3rd level, 25gp at 4th level, and 30gp at 5th level.

\paragraph*{Poisonous Innovation}
  At 1st level, you may use your personal labratory to study and experiment with poisons.
  When crafting a poison, you may choose to change some of its six attributes.
  These attributes are: Delivery method, status effect imposed, damage dealt, duration, delay, and difficulty class.
  Options for each attribute are given below, with the standard attribute listed first and italicized.

  You may change a number of attributes up to your Master of Weeds level + 1.
  
  

\paragraph*{Cautious Cook}
At 2nd level, you can spend equal time and material components to create an antidote specific to any poison you can create.
This antidote negates any pending or lingering effects and grants immunity to that poison for one hour.

Additionally, you can craft a generic Antitoxin, granting advantage on saving throws against poison for one hour.  (Refer to Chapter 5 of the Player's Handbook for details on Antitoxin.)

\paragraph*{Death Eater}
At 4th level, your constant exposure to poison has bolstered your resistance to it.
You gain resistence to poison damage and have advantage on saving throws against poison.

\paragraph*{Poison Savant}
You roll with advantage when creating poisons.










\paragraph*{Research Poisons}
Your understanding of poisons enables you to create a number of effects.

\begin{tabular}{ll}
{\it \bf Attribute} & {\bf Crafting Cost Impact} \\
{\bf Delivery Method} &\\
{\it Injury} & -- \\
Contact      & -- \\
Ingestion    & -- \\
Inhaling     & -- \\
&\\
&\\
{\bf Status Effect Imposed }&\\
{\it None}  &\\
Blinded     &\\
Deafened    &\\
Frightened  &\\
Paralyzed   &\\
Poisoned    &\\
Stunned     &\\
Unconscious &\\
&\\
&\\
{\bf Damage Dealt }&\\
{\it 1d4}  & +0 DC  \\
&\\
&\\
{\bf Duration }&\\
{\it 1 minute}  & +0 DC  \\
1 round   & -10 DC \\
1 hour    & +10 DC \\
1 day     & +20 DC \\
&\\
&\\
{\bf Delay }&\\
{\it No delay} &\\
1 round delay  &\\
1 minute delay &\\
1 hour delay   &\\
1 day delay    &\\
&\\
&\\
{\bf Save DC } &\\
{\it DC 10} &\\
&\\
&\\
\end{tabular}
\end{document}
